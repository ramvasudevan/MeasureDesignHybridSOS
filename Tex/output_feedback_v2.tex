\documentclass[dvipsnames]{article}
\usepackage[english]{babel}
\usepackage{amsmath,amssymb}
\usepackage[margin=1in]{geometry}
\usepackage{color}
\usepackage{amsthm}
\usepackage[hidelinks]{hyperref}
\usepackage{mathtools}
\usepackage{enumerate}
\usepackage{tcolorbox}
\usepackage{ragged2e}
\definecolor{mycolor}{rgb}{0.122, 0.435, 0.698}
%\newtcolorbox{mybox}{colback=red!5!white,colframe=red!75!black}

\newtcolorbox{mybox}{colback=mycolor!10}
\newtcolorbox{mybox_red}{colback=red!10}

\newtheorem{lemma}{Lemma}
\newtheorem{proposition}{Proposition}
\newtheorem{theorem}{Theorem}
\newtheorem{fact}{Fact}
\newtheorem{definition}{Definition}
\setlength{\parindent}{0in}
\setlength{\parskip}{.1in}
\providecommand{\supp}{\ensuremath \text{supp }}
\providecommand{\R}{\ensuremath \mathbb{R}}
\providecommand{\N}{\ensuremath \mathbb{N}}
\providecommand{\ip}[1]{\ensuremath \langle #1 \rangle}
\providecommand{\eq}[1]{\ensuremath\stackrel{#1}{=}}
\begin{document}
 \title{Output feedback control law synthesis for a class of polynomial systems}
  \date{}
  \maketitle
%  \justify
{\bf Full State-feedback : Primal problem} ({\bf P$_f$})
\begin{flalign*}
&\supp\,\mu_0(X)\\
&\text{subject to}\\
    &&\,\mathcal L_f'\mu+\mathcal L_g'(\sigma^+-\sigma^-)+\delta_0\times \mu_0=&\,\delta_T\times \mu_T\\
    &&\,[\sigma^+]_k+[\sigma^-]_k+[\hat\sigma]_k=&\,\mu&\,\forall k\in\{1,\,\ldots,\,m\}\\
    &&\,\mu_0+\hat \mu_0=&\,\lambda\\
    &&[\sigma^+]_k,\,[\sigma^-]_k,\,[\hat \sigma]_k\ge&\, 0&\,\forall k\in\{1,\,\ldots,\,m\}\\
    &&\mu,\,\mu_0,\,\hat\mu,\,\mu_T\ge&\, 0\\
    &&\supp(\mu_T)\subset X_T\\
    &&\supp(\mu_0)\subset X_1\times X_2\\
    &&\supp(\hat \mu_0)\subset X_1\times X_2\\
    &&\supp(\mu)\subset T\times X_1\times X_2
\end{flalign*}

{\bf Partial State-feedback : Primal problem} ({\bf P$_p$})
\par
Let $\Gamma_m=\N_m$ and $\Lambda=\{\bar\mu,\mu_0,\,\mu_T,\,\hat\mu_0,\,[\bar\sigma^+]_\Gamma,\,[\bar\sigma^-]_\Gamma,\,[\hat{\bar\sigma}]_\Gamma\}$
\begin{flalign*}
&\operatorname*{\text{supp}}_{\Lambda}\,\mu_0(X)\\
&\text{subject to}\\
    &&\,\mathcal L_f'\mu+\mathcal L_g'(\sigma^+-\sigma^-)+\delta_0\times \mu_0=&\,\delta_T\times \mu_T\\
    &&\,[\sigma^+]_k+[\sigma^-]_k+[\hat\sigma]_k=&\,\mu&\,\forall k\in\{1,\,\ldots,\,m\}\\
    && d\nu^*(x_2|t,x_1)d\bar\mu=&\,d\mu\\
    && d\nu^*(x_2|t,x_1)d[\bar\sigma^+]_k=&\,d[\sigma^+]_k&\,\forall k\in\{1,\,\ldots,\,m\}\\
    && d\nu^*(x_2|t,x_1)d[\bar\sigma^-]_k=&\,d[\sigma^-]_k&\,\forall k\in\{1,\,\ldots,\,m\}\\
    && d\nu^*(x_2|t,x_1)d[\hat{\bar\sigma}]_k=&\,d[\hat \sigma]_k&\,\forall k\in\{1,\,\ldots,\,m\}\\
    &&\,\mu_0+\hat \mu_0=&\,\lambda\\
    &&[\bar\sigma^+]_k,\,[\bar\sigma^-]_k,\,[\hat{\bar \sigma}]_k\ge&\, 0&\,\forall k\in\{1,\,\ldots,\,m\}\\
    &&\bar\mu,\,\mu_0,\,\hat\mu,\,\mu_T\ge&\, 0\\
    &&\supp(\mu_T)\subset X_T\\
    &&\supp(\mu_0)\subset X_1\times X_2\\
    &&\supp(\hat \mu_0)\subset X_1\times X_2\\
    &&\supp(\bar\mu)\subset T\times X_1\\
    &&\supp(\bar\sigma^\pm)\subset (T\times X_1)^m\\
    &&\supp(\hat{\bar\sigma})\subset (T\times X_1)^m
\end{flalign*}
where $\mu^*$ is the optimal solution to {\bf P$_f$} and $d\mu^*=d\nu^*(x_2|t,x_1)d\pi_{t,x_1}\mu^*(t,x_1)$.
\par
{\bf Partial State-feedback : Dual problem} ({\bf D$_p$})\\
In the dual formulation below, let $\phi\colon C_c(T\times X_1\times X_2)\rightarrow C_c(T\times X_1)$ be the operator representation of integrating-out the conditional $\nu^*(x_2|t,x_1)$; i.e.
\begin{flalign*}
  &&v(t,x_1,x_2)\mapsto&\,\int_{X_2}v(t,x_1,x_2)\,d\nu^*(x_2|t,x_1).\,&\forall v\in C_c(T\times X_1\times X_2)
\end{flalign*}
%Observe that $\phi$ is a linear function.\par%%
\underline{The dual:}
\begin{flalign*}
&\inf\,\ip{\lambda,w}\\
&\text{subject to}\\
    &&\phi\circ\mathcal L_fv+\sum_{k=1}^m \phi\circ[p]_k\,\le&0\\
    &&w- v(0,\cdot)-1\ge&\, 0\\
    &&v(T,\cdot)\ge &\,0\\
    &&w\ge&\, 0\\
    &&\phi\circ[p]_k\ge& \,0,\phantom{10}\phi\circ[p]_k\ge \,|\phi\circ[\mathcal L_gv]_k|&\,\forall k\in\{1,\,\ldots,\,m\}
\end{flalign*}
where $w\in C(X_1\times X_2),\,v\in C^1(T\times X_1\times X_2),\,p_k\in C(T\times X_1\times X_2),\,q\in C(T\times X_1\times X_2)$ and $`\circ'$ is used to denote the action of operator $\phi$ on a function.

\centering{\rule{.5\columnwidth}{1pt}}
%%%%%
\begin{definition}[{\cite[Definition 1.3.4]{bogachev_v1}}]
  A countably additive measure $\mu$ on a space $(X,\Sigma)$ is a probability measure if $\mu\ge 0$ and $\mu(X)=1$.
\end{definition}
%%%%%
\begin{definition}
  Suppose $\mu$ is a measure on the space $(X,\Sigma)$. The Jordan-Hahn decomposition of $\mu$ is unique and given by
  \begin{align*}
    \mu=\mu^+-\mu^-,
  \end{align*}
  where $\mu^+,\,\mu^-\ge 0$. The total variation of $\mu$ is defined as $|\mu|=\mu^++\mu^-$. Clearly, if $\mu\ge 0$, then $|\mu|=\mu$.
\end{definition}
%%%%%
\begin{fact}[{\cite[Theorem 10.4.14]{bogachev_v2}}](\textcolor{red}{Check!!})
  Let $\mu$ be an unsigned measure defined on $(X\times Y,\,\mathcal B)$ where $X\subset \R^n$, $Y\subset \R^m$, $\mathcal B$ is the Borel $\sigma$-algebra, and $\pi_{y}\mu$ be the projection (marginal) onto $Y$. Then, $\forall y\in Y$, there exists an unsigned measure $\mu(\cdot,y)$ on $\mathcal B$ such that
  \begin{align*}
    d\mu=d\nu(x_1|x_2)\,d\pi_{x_2}\mu.
  \end{align*}
\label{fact:unsigned_conditional}
\end{fact}
%%%%%
\begin{fact}[{\cite[Remark 10.4.4]{bogachev_v2}}](\textcolor{red}{Check!!})
  If $\sigma$ is a signed measure on $(X\times Y, \mathcal B)$ possessing as regular conditional measures, unsigned measures $\nu_\sigma(\cdot,\,y)$, $\forall y\in Y$, $|\sigma|$ admits $\nu_\sigma(\cdot,\,y)$ as its regular conditional measure.
  \par
  Conversely, if $\mu(\cdot,y)$ is the regular conditional measure for $\forall y\in Y$ of $|\mu|$ and it is unsigned, then the regular conditional measures of $\mu^+$ and $\mu^-$ are $\nu(\cdot,y)$.
\label{fact:variation}
\end{fact}
%%%%%
%%%%%
\begin{lemma}[Sufficiency]
\label{lemma:sufficiency}
    Let $\mu$ and $\sigma$ be radon measures on the product space $X_1\times X_2$ such that $\sigma \ll \mu$, where $X_1$ and $X_2$ are Polish spaces. Then each measure can be decomposed as follows
    \begin{align}
    \begin{aligned}
        d\mu=&\,d\nu_{\mu}(x_2|x_1)d\pi_{x_1}\mu(x_1)\\
        d\sigma=&\,d\nu_{\sigma}(x_2|x_1)d\pi_{x_1}\sigma(x_1).
        \end{aligned},
        \label{eqn:polish:decomp}
    \end{align}
    where $\pi_{x_1}\mu$ is the $x_1$ marginal of $\mu$. If $d\nu_\sigma(x_2|x_1)=d\nu_{\mu}(x_2|x_1)$, then the Radon-Nikodym derivative $\dfrac{d\sigma}{d\mu}$ is not a function of $x_2$.
    \par
    Additionally, if $\sigma$ and $\mu$ satisfy the above conditions with $\mu\ge 0$, $\exists\, \hat\sigma\ge 0$ such that
    \begin{align}
      \sigma+\hat\sigma=\mu\label{eqn:sufficiency:domination:equality}
    \end{align}
    and $d\hat\sigma=d\nu_\mu(x_2|x_1)\,d\hat{\bar\sigma}$. That is, the regular conditional of $\hat \sigma$ given $x_1$ is identical to that of $\mu$ and $\sigma$.
\end{lemma}
\begin{proof}
  That each measure can be decomposed as in Eqn.~(\ref{eqn:polish:decomp}) follows from a standard result in measure theory (\cite{bogachev_v2}) and hence we concentrate on the Radon-Nikodym derivative. Since $\sigma\ll \mu$, from the definition of marginals, $\pi_{x_1}\sigma\ll \pi_{x_1}\mu$; let $d\pi_{x_1}\sigma=\phi(x_1)\,d\pi_{x_1}\mu$.\par
  Consider an arbitrary test function $v(x_1,x_2)$; then the following equalities holds by definition.
  \begin{align*}
    g(x_1):=\int_{X_2}v(x_1,x_2)\,d\nu_{\sigma}(x_2|x_1)=\int_{X_2}v(x_1,x_2)\,d\nu_{\mu}(x_2|x_1)
  \end{align*}
  \begin{align*}
    \int_{X_1\times X_2} v(x_1,x_2)\,d\sigma=&\,\int_{X_1}g(x_1)\,d\pi_{x_1}\sigma(x_1)\\
    =&\,\int_{X_1\times X_2}\phi(x_1)\,g(x_1)\,d\pi_{x_1}\mu(x_1)\\
    =&\,\int_{X_1\times X_2} \phi(x_1)\,v(x_1,x_2)\,d\mu
  \end{align*}
  Thus $d\sigma=\phi(x_1)\,d\mu$ and the Radon-Nikodym derivative is not a function of $x_2$.
  \par
  The existence of $\hat\sigma \ge 0$ such that Eqn.~(\ref{eqn:sufficiency:domination:equality}) holds since $\sigma\ll\mu$. Additionally, as $\sigma$ and $\mu$ share the same conditional with respect to $x_1$, $\sigma$ admits the following decomposition.
  \begin{align*}
    d\hat \sigma=&\,d\nu(x_2|x_1)\,d(\pi_{x_1}\mu-\pi_{x_1}\sigma)(x_1)\\
    =&\,d\nu(x_2|x_1)\,d\hat{\bar\sigma}(x_1)
  \end{align*}
\end{proof}
%%%%

%\flushleft

%%%%
\begin{lemma}
    The solution to the primal problem characterizes the BRS when using partial state-feedback control.
\label{lemma:BRS}
\end{lemma}
\begin{proof}
In this proof, we consider the special case of single input control; a more general version follows naturally. For convenience, denote by $\mathcal U_s$ and $\mathcal U_p$ respectively, the set of all admissible state-feedback and partial-state-feedback control laws; by definition, $\mathcal U_p\subseteq \mathcal U_s$. Similarly, let $\mathcal X_s$ and $\mathcal X_p$ represent the collection of admissible state trajectories; $\mathcal X_p\subseteq \mathcal X_s$.\par
By definition, for any initial condition $x_0\in X_0^o$, there exists a partial state-feedback control law $u(t,x_1)$ such that the resulting state trajectory is admissible. Thus, for any initial measure $\mu_0$, there exist measures $\mu,\,\mu_T,\sigma$ which satisfy the following constraints
\begin{flalign*}
  &&\,\delta_{T}\times \mu_T=&\,\delta_{0}\times\mu_0+\mathcal L_{f}'\mu+\mathcal L_g'\sigma&
\end{flalign*}
such that $\mu_0,\,\mu_T,\,\mu\ge 0,\,\supp(\mu_0)\subset\mathcal X,\,\supp(\mu_T)\subset X_T$ and $\supp(\mu)\subset T\times X_1\times X_2$. This assertion follows from applying the result in \cite[Lemma 1]{Henrion} to the state-feedback problem posed in \cite{Anirudha} for any feasible trajectory and reviewing the relation between $\mathcal U_s$ and $\mathcal U_p$, and $\mathcal X_s$ and $\mathcal X_p$.\par
 Since the Euclidean space is Polish and hence Souslin, and separable, it follows that $\mu$ can be decomposed as follows \cite[Corollary 10.4.13]{bogachev_v2}
\begin{align*}
    d\mu=&\,d\nu(x_2|t,x_1)\,d\pi_{t,x_1}\mu(t,x_1).
\end{align*}
Now, since the occupation measure can be interpreted as the time spent in the region of the product space, naturally, the $(t,x_1)$ marginal ($\pi_{t,x_1}\mu$), is the total time spent by all feasible state trajectories in a slice of the product space; and $\nu(x_2|t,x_1)$ can be interpreted as measure of how well the trajectories are distributed along the $x_2$ direction at every time instant (\textcolor{red}{??}). Naturally, if $\nu^*(A|t,x_1)=0$, then $\nu(A|t,x_1)=0$ $\forall (t,x_1)\in T\times X_1$ ($\because\mathcal X_p\subseteq\mathcal X_s$); and hence $\nu(\cdot|t,x_1)\ll \nu^*(\cdot|t,x_1)$. Thus, using \cite[Theorem 58]{dellachere}, $\exists \phi(t,x_1)$, a $\pi_{t,x_1}\mu$ measurable function such that
\begin{align*}
\label{eq:domination:conditional:2}
   d\mu =&\,\phi(t,x_1)\,d\nu^*(x_2|t,x_1)\,d\pi_{t,x_1}\mu(t,x_1),\\\nonumber
    =&\,d\nu^*(x_2|t,x_1)\,d\bar\mu(t,x_1),
\end{align*}
where the final equality is obtained by appropriately defining $\bar\mu$.
\par
Since $\sigma$ and $\mu$ are solutions to the Liouville equation with R-N derivative \mbox{$u=u(t,x_1)$}, the following relations hold
\begin{align*}
  d\sigma=&\,u\,d\mu,\\
  =&\,ud\nu^*(x_2|t,x_1)\,d\bar\mu(t,x_1),\\
  =&\,d\nu^*(x_2|t,x_1)\,d\bar\sigma(t,x_1),
\end{align*}
with $d\bar \sigma=u\,d\bar\mu$.
\par
Define $\sigma^+,\,\sigma^-\ge 0$ such that the following holds (Jordan decomposition)
\begin{align*}
  \sigma=\sigma^+-\sigma^-.
\end{align*}
The regular conditional measures of $\sigma^+$ and $\sigma^-$, $\forall y\in Y$ are both $\nu^*(\cdot|t,x_1)$ (Facts \ref{fact:unsigned_conditional}~\&~\ref{fact:variation}). Consequently, $\sigma^\pm$ can be decomposed as follows
\begin{align*}
  d\sigma^+=&\,d\nu^*(x_2|t,x_1)\,d\bar\sigma^+(t,x_1)\\
  d\sigma^-=&\,d\nu^*(x_2|t,x_1)\,d\bar\sigma^-(t,x_1).
\end{align*}
Finally, to see that $\hat\sigma$ shares the same conditional as $\mu$, use Fact~\ref{fact:variation} and Lemma~\ref{lemma:sufficiency}.
\par
Thus, it follows that for every initial condition in the BRS, there exists a feasible solution to {\bf P$_p$} and hence the optimization problem {\bf P$_p$} is stronger that any other any problem that exactly identifies the BRS. Thus if $q^*$ is the optimal value of the cost, then $q^*\ge \lambda(X_0^o)$.
\par
Next, we show that $q^*\le \lambda(X_0^o)$ by contradiction. Let $(\mu_0,\,\mu_T,\,\mu,\,\sigma^+,\,\sigma^-,\,\hat\sigma)$ be a feasible solution to {\bf P$_p$} and suppose $\lambda\,(A:=\supp(\mu_0)\backslash X_0^o)\ne 0$. By definition, $\exists u=u(t,x_1)$ such that $d\sigma=u\,d\mu$; re-define the dynamics of the system by subsuming this control law into the drift vector field; i.e.
\begin{align*}
  \dot x=\bar f(t,x_1,x_2)=f(t,x_1,x_2)+u(t,x_1)g(t,x_1,x_2).
\end{align*}
It is evident that the tuple $(\mu_0,\,\mu,\,\mu_T)$ is a feasible solution to the optimization problem
\begin{align*}
  \delta_0\times \mu_0=\delta_T\times \mu_T+\mathcal L_{\bar f}'\mu.
\end{align*}
Using \cite[Lemma 3]{Henrion}, it then follows that there exists a family of admissible trajectories emanating from $\supp(\mu_0)$ and terminating in $\supp(\mu_T)$. This is a contradiction since trajectories starting from $\supp(\mu)\backslash X_0^o$ cannot be admissible; i.e. $\lambda(X_0^o)\ge \lambda(\supp(\mu_0))$. Thus, $\lambda(X^o_0)=\lambda(\supp (\mu_0))$.
\end{proof}

%%%%%%
\justify
The next Lemma is a variation of \cite[Lemma 2]{Henrion} in that it establishes that the BRS can be identified as the interior of the super-level set $\{(x_1,x_2)\mid w(x_1,x_2)\ge 1\}$.

\begin{lemma}
Let $(v,w,p_{1,\ldots,m})$ be a tuple of feasible solutions to the dual {\bf D$_p$}. Then $v(0,\cdot,\cdot)\ge 0$ and $w\ge 1$ on $X_0\,(:=\supp (\mu_0))$.
\end{lemma}
\begin{proof}
    By the fundamental theorem of calculus and the constraints of {\bf D$_p$},
    \begin{align*}
      0\le v(T,x(T))=&\,v(0,x(0))+\int_{0}^T \mathcal L_fv+\mathcal L_gv\,u(t)\, dt\\
      =&\,v(0,x(0))+\int_{T\times X_1\times X_2} \left(\mathcal L_fv+\mathcal L_gv\,u(t,x_1,x_2)\, \right)d\mu\\
      =&\,v(0,x(0))+\int_{T\times X_1} \left(\phi\circ\mathcal L_fv+\phi \circ (\mathcal L_gv\,u(t,x_1,x_2))\,\right) d\bar\mu\\
      =&\,v(0,x(0))+\int_{T\times X_1} \left(\phi\circ\mathcal L_fv+\sum_{i=1}^m\phi\circ [p]_i+\phi \circ (\mathcal L_gv\,u(t,x))-\sum_{i=1}^m\phi\circ [p]_i\, \right)d\bar\mu\\
      \le&\,v(0,x(0))\le w(x(0))-1
    \end{align*}
    where $x(t)=(x_1(t),x_2(t))$.
\end{proof}
%%%%%%
\begin{mybox_red}
{\bf Notes / todo}
\begin{enumerate}
    \item Check to ensure that all text in red make sense.
    \item Why is it okay to optimize over just $\bar\mu\in \mathcal M(T\times X_1)$? Use result in \cite[Theorem 10.7.2]{bogachev_v2}.
    \item Using the regular conditional from an unsigned measure and a positive `marginal', we can construct another unsigned measure?
    \item Evaluating output-feedback controllability using inner approximations
    \item Change notations to ensure there is no confusions between the final time $T$ and the interval $T=[0,\,T]$.
    \item Having changed the primal problem formulation, need to show that the $\hat\sigma$ has the same conditional as $\mu$. (Lemma~\ref{lemma:BRS})
    \item Check to see if the assumption of controllability is required and the kind of dynamical systems can be entertained \cite[Assumption 2]{Henrion}.
\end{enumerate}
\end{mybox_red}

\begin{thebibliography}{9}
%  \bibitem{Folland}
%  Folland, Gerald B. ``Real Analysis: Modern Techniques and Their Applications, John Willey and Sons.'' Inc., New York (1999).
%  \bibitem{Dubins}
%  Dubins, Lester, and David Freedman. ``Measurable sets of measures." Pacific Journal of Mathematics 14.4 (1964): 1211-1222.
  \bibitem{Henrion}
  Henrion, D.; Korda, M., ``Convex Computation of the Region of Attraction of Polynomial Control Systems," Automatic Control, IEEE Transactions on , vol.59, no.2, pp.297,312, Feb. 2014
    doi: 10.1109/TAC.2013.2283095
\bibitem{Anirudha}
Majumdar, Anirudha, et al. ``Convex optimization of nonlinear feedback controllers via occupation measures." The International Journal of Robotics Research (2014): 0278364914528059.
\bibitem{bogachev_v1}
Bogachev, Vladimir I. Measure theory. Vol. 1. Springer Science \& Business Media, 2007.
\bibitem{bogachev_v2}
Bogachev, Vladimir I. Measure theory. Vol. 2. Springer Science \& Business Media, 2007.
\bibitem{dellachere}
Dellacherie, Claude, and Paul-André Meyer. "Probabilities and potential. B, volume 72 of North-Holland Mathematics Studies." (1982).
\end{thebibliography}
\end{document}
